\documentclass[aps,superscriptaddress]{revtex4}
\usepackage[utf8]{inputenc}
\usepackage{graphicx}
\newcommand{\tp}{t_{\scriptstyle \rm p}}
\newcommand{\tpum}{t_{\scriptstyle{\rm p}_1}}
\newcommand{\Tf}{T_{\scriptstyle\rm F}}


\begin{document}
\title{Phase transitions and geometry in spin models: the cluster size heterogeneity}

\author{*Artur U. Fröhlich}
\email{artur.uhlik@gmail.com}
\affiliation{Instituto de F\'\i sica, Universidade Federal do Rio Grande do Sul, CP 15051, 91501-970, Porto Alegre RS, Brazil}
\author{Renan A.L. Almeida}
\email{ra.lisboaalmeida@gmail.com}
\affiliation{Instituto de F\'\i sica, Universidade Federal do Rio Grande do Sul, CP 15051, 91501-970, Porto Alegre RS, Brazil}\author{Jeferson J. Arenzon}
\email{arenzon@if.ufrgs.br}
\affiliation{Instituto de F\'\i sica, Universidade Federal do Rio Grande do Sul, CP 15051, 91501-970, Porto Alegre RS, Brazil}
\affiliation{Instituto Nacional de Ciência e Tecnologia - Sistemas Complexos, Rio de Janeiro RJ, Brazil}



\vskip 1\baselineskip
\begin{abstract}
\large
The geometrical analysis of thermal phase transitions has unveiled a series of new features that helped both the understanding of experimental results and the introduction of new simulation algorithms.
The cluster size heterogeneity $H$ was introduced by Lee \textit{et al.}~\cite{LeKiPa11}, and has been successfully  applied to domains of parallel spins in the Ising model~\cite{JoYiBaKi12,RoOlAr15,AzAlOlAr22} and further generalized~\cite{MaAzArCo21}. It is a measure of how different the sizes of the clusters that form the system are. 
The value of $H$ is close to unity both at very low and very high temperatures where there is a single cluster or many small, almost identical clusters, respectively.
Away from these limits, it presents a peak very close to the critical transition $T_c$, both in 2d and 3d.
Interestingly, $H$ has a second peak at a temperature well above the transition, and the height of this peak seems rather independent of the model~\cite{AzAlOlAr22}.  
The location of this second peak, extrapolated to the thermodynamical transition, was shown in Ref.~\cite{RoOlAr15} to be consistent with $T_c$.
However, the range of system sizes considered, and therefore the precision of this conclusion, was rather limited.
Our objective here is to study larger systems and check whether this convergence indeed goes to $T_c$ or, otherwise, may indicate the presence of a higher-order transition, as proposed by Sitarachu and Bachmann~\cite{SiBa22}, above $T_c$.

\end{abstract}
\maketitle


\bibliographystyle{apsrev}   
\bibliography{hetbib}
\end{document}           

